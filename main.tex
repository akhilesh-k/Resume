% Author: Akhilesh Kumar
% Compiler: XeLatex
% Configuration:

%Paper size and font
\documentclass[a4paper,10pt]{extarticle}
% For loading fonts
\usepackage{fontspec} 
\defaultfontfeatures{Mapping=tex-text}
% Used Ubuntu Fonts
\setmainfont[Path = ./Ubuntu/,  
 Extension = .ttf,
 UprightFont = *-Regular,
 BoldFont = *-Bold,
 ItalicFont = *-Italic,
 SmallCapsFont = *-Medium]
{Ubuntu}
\usepackage{color}
\definecolor{primary}{RGB}{130, 2, 99}
\definecolor{secondary}{RGB}{251, 139, 36}
\usepackage{xunicode,xltxtra,url,parskip}
\usepackage[usenames,dvipsnames]{xcolor}
\usepackage{geometry}
\geometry{a4paper,margin=0.40cm}
% Link configuration
\usepackage{hyperref}
% Link color
\definecolor{linkcolour}{rgb}{0.3,0.3,0.3} 
% Set link colors throughout the document
\hypersetup{colorlinks,breaklinks,urlcolor=linkcolour,linkcolor=linkcolour} 

\usepackage{titlesec}
\titleformat{\section}{\large\scshape\raggedright}{}{0em}{}[\titlerule]
\titlespacing{\section}{0pt}{0pt}{0pt}
\usepackage{multicol}
\setlength{\columnsep}{0cm}
\usepackage{tabularx}
\usepackage{textcomp}
\usepackage{fontawesome}
\usepackage{enumitem}
\setlist[description]{%
  topsep=10pt,
  itemsep=1pt,
}

\def\arraystretch{1}
\renewcommand{\baselinestretch}{1.1}

\begin{document}

\pagestyle{empty}

% Name and Contact
\begin{multicols}{3}
\normalsize  \faGlobe\ {\href{http://akhileshkumar.me/}{\  akhileshkumar.me}}\\
\normalsize \faGithub\ {\href{https://github.com/akhilesh-k}{\  akhilesh-k}}\\
\normalsize  \faLinkedinSquare\ {\href{https://www.linkedin.com/in/akhilesh-k}{\  akhilesh-k}}\\
\columnbreak
\normalsize\par{\centering{\huge\textsc{\textcolor{primary}{Akhilesh} Kumar}}\par}
\par{\centering\normalsize {\textsc{Shastri Bhawan, JUIT Waknaghat, HP, India - 173234}}\hfill\par}
\columnbreak
\raggedright\hfill\normalsize \faEnvelope\ {\href{mailto:akhileshk.juit@gmail.com}{\  akhileshk.juit@gmail.com}}\\
\raggedright\hfill{\faPhone\ \  +91-9829754634}\\
\end{multicols}

% Education
\vspace{-0.6cm}
\section{\textcolor{primary}{Academics}}
\begin{tabular}{r|p{17.5cm}}	
2016-2020 & B.Tech in \textbf{Electronics and Communication Engineering}\hfill\textsc{GPA}: 7.0/10.0\\
\textsc{(Expected)}&\textbf{Jaypee University of Information Technology}, Waknaghat\\

%&\textbf{Coursework: }{Programming and Data Structures, Algorithms-I \& II, Software Engineering, Compilers, Switching Circuits, Operating Systems, Computer Networks, Information Retrieval, Database Management Systems, Theory of Computation, Machine Learning, Image Processing, Advanced Graph Theory}
\end{tabular}
%Skills
\section{\textcolor{primary}{Technical Skills}}

\begin{tabular}{r|p{15cm}}
\textsc{Programming} & C++, Python, Javascript\\
%& \textit{Competent:} Java, Golang, PHP \\
\textsc{Libraries / Frameworks} & Node.js, ReactJS, Tensorflow, Keras, OpenCV, ROS\\
\textsc{Databases} & MySQL, PostgreSQL, MongoDB\\
\textsc{Systems / Platforms} & Git, AWS, Docker, Heroku, Linux\\
\end{tabular}

\section{\textcolor{primary}{Experience}}

\begin{tabularx}{\linewidth}{ l | X }

\textsc{Jul 18} & \textbf{Machine Learning Intern}\hfill\textbf{USHR, India}\\
\textsc{May 18}& {- Interpret data on price, yield, stability, future investment-risk trends, economic influences, and other factors affecting investment programs using Data Analytics.}\\
& {- Worked on Data Scrapping, Fuzzing, Preprocessing on Documents and Setting up a multi-label Classifier.}\\
\end{tabularx}

% Academic Projects
\section{\textcolor{primary}{Academic Projects}}
\vspace{-0.6cm}
\begin{tabular}{p{19.7cm}}

\begin{description}[style=nextline, font=$\bullet$\hspace{2mm}\normalsize]

 \item [Implementation of Digital Filter in Real Time using DSP C2000 Launchpad] Designed and implement FIR and IIR digital filters in real time on the DSP C2000 LaunchPad using audible signals and tones acquired in 4 different channels and the option to mix these signals individually as part of DSP Course.
 
  \item[Microwave circuit optimization for impedance matching.] Wrote Object oriented Python Script for Optimized Impedance matching in Microwave RF circuit.
 
 \item[Data Communications in Python] Wrote multiple python script for explaining freqency-amplitude relationship for various Data Communication types.
 

 \item[Response Detection in Verilog] Built an end to end system for early signal detection in Verilog.Computed D to Q delay and clock to Q delay for determining response delays. The application is Buzzer type Quiz system devices.
\end{description}
\end{tabular}

% Projects
\section{\textcolor{primary}{Research Projects}}
\vspace{-0.6cm}
\begin{tabular}{p{19.7cm}}

\begin{description}[style=nextline, font=$\bullet$\hspace{2mm}\normalsize]

 \item [Standalone Driving Assistant Unit]Developed a dash camera based standalone pipeline with functionalities of Lane Departure Warning, Forward Collision Warning and Tailgate warning.
 
  \item[Curved lane lines detection using HSV filtering and sliding window search method.]Implemented a Robust Curved lane detection pipeline built on top of Python. Based on HSV filtering and Sliding window search algorithm with overlay of detected road.
 

 \item[Labeling pixels of a road in images using a FCN using Semantic Segmentation approach]Built a Fully Convolutional Network (FCN) that could label the individual pixels of an image as road or not road. The FCN was trained to recognize two classes: road and not road. The final network was trained for 10 epochs using a batch size of 4 and is based on the FCN-8 architecture built using the VGG network and trained on the KITTI Dataset.
\end{description}
\end{tabular}
\vspace{-0.3cm}
\section{\textcolor{primary}{Hackathons \& Competitions}}

\begin{tabularx}{\linewidth}{ l | X }
\textsc{Current} & \textbf{Pedestrian Safety Device}\hfill\textbf{Smart India Hackathon '19}\\
& {- Built a Pedestrian Detection Pipeline using INRIA dataset and YOLO model with Darknet framework. Comparatively analyzed the efficiency of alert trigger with INRIA and DALIMAR datasets.}\\
\multicolumn{2}{c}{} \\
\textsc{Nov 17} & \textbf{IoT based Pollution Monitoring and Waste Management for smart cities}\hfill\textbf{Smart Cities Hackathon-'18}\\
& {- Won 2nd Prize for building a Smart City smart waste management dashboard with various utilities created for municipalities. The dashboard was built with a NodeJS backend and had several utilities including plots, optimal routes, grievance portal and municipal vehicle finder to name a few.}\\
\multicolumn{2}{c}{} \\

\textsc{Feb 17} & \textbf{Underwater Glider for Real Time Mapping with SensorTag IoT System}\hfill\textbf{Murious 2017}\\
& {- Accomplished automated glider controlled movement with a ballast system.
Developed obstacle-avoiding feature and
Interfaced TI CC2650STK SensorTag with Raspberry Pi to retrieve data in real time.}\\
\multicolumn{2}{c}{} \\




\end{tabularx}


% Coursework

 \section{\textcolor{primary}{Academic Coursework}\hfill\small\textsc{(T)heory and (L)aboratory}}}
 

 \begin{multicols}{2}
 - Programming and Data Structures (T/L) \\
 - Discrete Mathematics (T)\\
 - Advanced Calculus (T)\\
 - Probability, Statistics \& Random Processes (T) \\
 - Signals and Systems (T/L) \\
 - Electrical Circuits (T/L) \\
 - VLSI Technology and Applications (T/L)\\
 - Object Oriented Systems and Programming (T/L) \\
 - Electromagnetic Engineering (T/L) \\
 - Analogue Electronics (T/L) \\
 - Linear Integrated Circuits (T/L) \\
 - Digital Electronics (T/L) \\
 - Analogue \& Digital Communications (T/L) \\
 - Digital Signal Processing (T\L) \\
 - Microprocessor and Controllers (T\L) \\
 - Telecommunication Networks (T\L)\\
 - Theory and Applications of Control System (T/L)\\
 - Microwave Devices & Antenna Design (T/L)
 \end{multicols}
 {\itshape{Currently Studying: *}}\\
% Certification

\vspace{-0.3cm}
\section{\textcolor{primary}{Certification}}

\begin{tabularx}{\linewidth}{ l | X }
\textsc{Current} & \textbf{Self Driving Car Engineer Nanodegree}\hfill\textbf{Udacity}\\
& {Learned Computer Vision, Deep Learning, and Sensor Fusion, Localization, Path Planning, Control, and System Integration with 11 projects in the Nanpdegree course.}\\
\multicolumn{2}{c}{} \\
\textsc{Dec 18} & \textbf{Deep Learning Specialization}\hfill\textbf{Coursera}\\
& {Learned basics of Deep Learning, Hyperparameter tuning, Regularization, Optimization, Convolutional Neural Networks, Building RNNs \& it's variants such as GRUs and LSTMs. Built projects on image and video recognition, classification and annotation. Various object detection techniques, motion estimation \& object tracking, human action recognition, and image stylization, editing and generation}\\
\multicolumn{2}{c}{} \\

\textsc{Mar 18} & \textbf{Robotics Specialization}\hfill\textbf{Coursera}\\
& {Completed courses on Aerial Robotics, Computational Motion Planning, Mobility, Perception and Estimation and Learning
Learned simulation, Path Planning, Sensor calibration, Designing of control algorithms and Extended Kalman filters to navigate autonomously
through designed environment}\\
\multicolumn{2}{c}{} \\

\textsc{Sept 17} & \textbf{Machine Learning}\hfill\textbf{Coursera}\\
& {Learned various algorithms for the foundation of Machine Learning and implemented on octave. Completed Spam Classifier and Hand written digit recogniser Project in this Course.}\\
\multicolumn{2}{c}{} \\




\end{tabularx}
% Position of Responsibility
\vspace{0.2cm}
\section{\textcolor{primary}{Extracurricular \& Leadership}}

\begin{tabular}{r|p{17.5cm}}
\textsc{Current} & \textbf{Founder \& Maintainer}, MetaJUIT Wiki\\
& {Responsible for fostering participation, promoting the growth of the group and maintaining the following open-source projects: metaqp, metaYP and metaImplode}\\
\textsc{Current} & \textbf{Vice Chairperson}, ACM Student Chapter JUIT \\
& {Responsible for forming event policies, Administratio and managing Robotics \& AI Projects in ACM Student Chapter of JUIT.}\\
\textsc{Current} &
\textbf{Coordinator}, JYC Media \& Publicity Committee \\

& {Leading a group of 45 students in areas of Digital Marketing (SMM, Email-Campaign), Graphic Designing/Video Editing}\\
\textsc{Current} & \textbf{Overall Coordinator}, TIEDC | E-Cell of JUIT\\
& {Actively building a vibrant startup ecosystems in Himachal Pradesh. I managed and coordinated with a team of 40 volunteers to organize Techstars Startupweekend Solan.}\\
\textsc{Sept 18} & \textbf{Organizer}, Techstars Startupweekend\\
& {Organized Techstars Startupweekend powered by Google for Entreprenuership}\\


\end{tabular}

% Honor and awards
\section{\textcolor{primary}{Honours \& Awards}}

\begin{multicols}{2}
- World Rank 37, BrainWaves 2017-18, (ML Contest)\\
 - Finalist, National Entrepreneurship Challenge (IIT B)
 \end{multicols}
% Achievement
\section{\textcolor{primary}{Scholastic Achievements}}

\begin{multicols}{2}
- Secured 97.11 percentile in JEE Advanced 2015 \\
- A+ in all the Lab Courses\\
 - Secured 95.33 percentile in JEE Mains 2015 
 
 \end{multicols}

%\new page

\end{document}
