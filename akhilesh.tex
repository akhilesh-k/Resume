
% Build using XeLaTex

\documentclass[letterpaper]{akhilesh}
\begin{document}

\namesection{Akhilesh}{Kumar}{ \urlstyle{same}\href{http://akhilesh-k.github.io}{http://www.akhileshk.in} \\
\href{mailto:akhilesh\_k@outlook.com}{akhilesh\_k@outlook.com} | +91-9829754634 \\
14H D/45 Azad Bhawan, JUIT Waknaghat, Himachal Pradesh-173234
}

%     COLUMN ONE

\begin{minipage}[t]{0.39\textwidth} 

%     EDUCATION

\section{Education} 

\subsection{Jaypee University Of Information Technology\ Waknaghat}
\descript{B.Tech Electronics and Communication Engineering\\Sophomore}\\
\location{Expected June 2020\\ CGPA: 6.6/10.0}
\sectionsep


\subsection{C.G. School, Bhabua}
\descript{High School}\\
\location{Grad. April 2015 | Bhabua, India \\Percentage: 72.4\%}
\sectionsep

\subsection{D.A.V. Public School}
\descript{Certificate of Merit}\\
\location{Grad. April 2013|  Sasaram, India\\CGPA: 10.0/10.0 }
\sectionsep

%     LINKS

\section{Links} 
Github: \href{https://github.com/akhilesh-k}{\custombold{akhilesh-k}} \\
LinkedIn:   \href{https://www.linkedin.com/in/akhilesh-k}{\custombold{akhilesh-k}} \\
\sectionsep

%     COURSEWORK

\section{Coursework}
\custombold{Programming \& Data Structures}\\
\custombold{Discrete Mathematics} \\
\custombold{Advanced Calculus} \\
\custombold{Robotics Fudamentals (edX)}\\
\custombold{Introduction to MATLAB (edX)}\\
\custombold{Computational Motion Planning (Coursera)}\\
\custombold{Machine Learning (Coursera)}\\
\href{https://www.coursera.org/account/accomplishments/certificate/QS5GBKTUNU2G}{\custombold{Aerial Robotics (coursera)}}
\sectionsep


%     SKILLS

\section{Technical Expertise}
\descript{Hardware}\\
ATmega \textbullet{} Raspberry Pi \textbullet{} TiLaunchpadsva C\\
%\sectionsep
\descript{Software}\\
Eagle \textbullet{} LabView \textbullet{} SolidWorks \textbullet{} Gazebo \textbullet{} Atmel Studio\\
%\sectionsep
\descript{Languages}\\
 C  \textbullet{} Python \textbullet{} C++ \textbullet{}   \LaTeX  \textbullet{} MATLAB\\BASH \textbullet{} Embedded C\\
%\sectionsep
\descript{Systems}\\
Git \textbullet{} OpenCV \textbullet{} IoT \textbullet{} ROS\\
\sectionsep



\section{Extra-Curriculars}  
Elocution \textbullet{} Dramatics \textbullet{} Basket Ball\\
\sectionsep

%     COLUMN TWO

\end{minipage} 
\hfill
\begin{minipage}[t]{0.599\textwidth} 


%     EXPERIENCE

\section{Research Experience/ Projects}

\runsubsection{\href{https://github.com/akhilesh-k/Wat-watcher/}{Under Water Glider for real time mapping with sensortag IoT system}}
\descript{| ACM-Electronics Team}\location{| Dec 2016 – Jan 2017}
\vspace{\topsep}
\begin{tightemize}
\item Accomplished automated movement of the glider controlled with a ballast system.
\ Developed obstacle-avoiding feature and algorithm for mapping of environment using MATLAB.
\item Used Sensor Tag TI CC2650 to pair it with Raspberry Pi device and retrieve data in real time.
\end{tightemize}
\sectionsep

\runsubsection{\href{https://github.com/akhilesh-k/CityIoTary}{IoT based pollution monitoring and waste management for smart cities}}
\descript{| ACM Electronics Team}\location{| Mar 2017 – Present}

\begin{tightemize}
\item Accomplished communication between Dustbins acreoss the city and server using Xbee modules with Raspberry Pi for localization.
\item Conceptualized model of routes optimization for waste pickups using Google Maps API.
\end{tightemize}
\sectionsep

\runsubsection{\href{https://github.com/akhilesh-k/Prodigy_Bot}{Motion Imitating and Path Replicating Robot enabled with mapping}}
\descript{| Project Leader }\location{| Mar 2017-Present}
\begin{tightemize}
\item Bot based on Arduino and interfaced with Raspberry Pi 
\item Uses Pi cam as input to handle control \& object recognition using OpenCV \& implemented python Client for Real time video streaming.
\end{tightemize}
\sectionsep

\runsubsection{\href{https://github.com/akhilesh-k/Intelligent-Alarm}{Intelligent Alarm system with Android UI}}
\descript{| ACM-Electronics Team}\location{| Nov 2016 - Dec 2016}
\begin{tightemize}
\item Pattern recognizing alarm based on Arduino which can set on from an android UI.
\end{tightemize}
\sectionsep

\runsubsection{\href{https://github.com/akhilesh-k/Mail-Notifier}{Mail notification system}}
\descript{}\location{| Feb 2017}
\begin{tightemize}
\item Built on Python Tkinter and serverside on Flask 
\end{tightemize}
\sectionsep

\runsubsection{\href{https://github.com/akhilesh-k/Stupid_AI}{PERSONAL ASSISTANT}}
\descript{}\location{| Mar 2017}
\begin{tightemize}
\item A rudimentary level AI Personal Assistant
\item Built on Python using Speech Recognition \& AIML
\item Developing GUI for the Virtual Assistant.
\end{tightemize}
\sectionsep

\section{Positions of responsibility}
\runsubsection{Technical Member}
\descript{| E-CELL -JUIT}
\begin{tightemize}
\item Member of University's E-CELL.
\end{tightemize}
\runsubsection{Organizing Member}
\descript{| IPR CELL- JUIT}
\begin{tightemize}
\item Organized Workshop on Patent filing
\end{tightemize}
\runsubsection{Organizing Committee}
\descript{| Technical Team}
\location{| February 2016 }
\begin{tightemize}
\item As part of the university’s official Annual Techfest- Murious XI
\end{tightemize}
\sectionsep 


\section{Co-Curricular Accolades}
\begin{tabular}{rl}
March 2017 & Participated in National Level Street Play-  Manthan\\
February 2017 & 2$^{nd}$ in University's Dramatics Competition- Halla Bol\\
\end{tabular}


\end{minipage} 
\end{document}  \documentclass[]{article}