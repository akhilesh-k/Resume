\cvsection{Projects}

\begin{cventries}
\cventry
    {Aerial and Underwater Robotics Society }
    {\href{https://github.com/akhilesh-k/Lane-and-Vehicles-Detection}{Real time Lane and Vehicles Detection}}
    {}
    {Sept 2017 - Present}
    {
      \begin{cvitems}
        \item {A computer vision software pipeline built on top of Python to identify vehicles in a video.}
        \item{Computes the camera calibration matrix and distortion coefficients for  distortion correction to raw images.}
        \item {Uses color transforms, gradients, Sobel, HOG feature extraction on a labeled training set of images, Vehicles classifier and Linear SVM classifier.}
        \item{Works as a pipeline on a video stream to create a heat map of recurring detection frame by frame to reject outliers and follow detected vehicles and etermine the curvature of the lane and vehicle position with respect to the center.}
      \end{cvitems}
    }
\cventry
    {ACM Electronics Team}
    {\href{https://github.com/akhilesh-k/BookWorm}{Ebook to Audio convertor using NLP and Google Speech}}
    {}
    {Sept 2017}
    {
      \begin{cvitems}
        \item {Developed an ebook to audio convertor using python.}
        \item{Implemented NLP for summarizing the stories, enabled to save audio outputs at mp4 locally, saves summaries in pdf or txt formats.}
        \item {Used Tkinter to develope GUI for the application.}
        \item{Won the Runners up appreciation prize at Hacksprint 2.0 at UIET, Chandigarh.}
      \end{cvitems}
    }
%\cventry
 %   {Mentor: \href{http://www.juit.ac.in/faculty.php?id=361&dep=ece&page=2}{Prof. Mohit Garg}}
  %  {Farming Robot: AGROBOT}
   % {}
    %{Sept 2017 - Present}
    %{
     % \begin{cvitems}
      %  \item {It is a multi function robot with autonomous grabber for weeding and seeding and Analysis.}
       % \item {A mobile robot to perform automatic seeding and moistening the soil after the seeding at a defined space interval using Manipulators.}
        %\item{It has a Navigation system, Data Analysis processing running in back end, Equipped with camera at different position for monitoring and decision making}
%      \end{cvitems}
 % \cventry
 %   {Mentor: \href{http://www.juit.ac.in/faculty.php?id=348&dep=ece&page=1}{Prof. Sunildatt Sharma}}
  %  {Determining the real time concentration of gases using Microband Antena.}
   %{}
    %{August 2017- Present}
    %{
     % \begin{cvitems}
      %  \item {Used Signal processing package for Octave to determine the real time concentration of gases in Atmosphere using an atena.}
       % \item {Used H\_FCM clustering algorithm to predict concentration for some know datasets.}
     %\end{cvitems}
    %}
 
     \cventry
    {ACM Electronics Team}
    {\href{https://github.com/akhilesh-k/CityIoTary}{IoT based Pollution Monitoring and Waste Management for smart cities}}
    {}
    {May 2017 - Jun 2017}
    {
      \begin{cvitems}
        \item {Established communication between dustbins \& Municipalities across the city with server on web using existing network.}
        \item {Conceptualized the Route Optimization using Google maps. Used python Requests library for sending coordinates stored.}
        \item {Uses Arduino, JS, Google Maps API and Backend of program runs on flask. Won 3rd Prize in Smart City Hackathon}
      \end{cvitems}
    }
    \cventry
    {ACM Electronics Team}
    {\href{https://github.com/akhilesh-k/Prodigy_Bot}{Motion Imitating and Path Replicating Robot}}
    {}
    {Mar 2017 - Apr 2017}
    {
      \begin{cvitems}
        \item {Arduino based Bot interfaced with Raspberry Pi capable of imitaing paths directed using aprilTags.}
        \item {Bot uses camera for input to handle controls using OpenCV. Developed python client for real time video stream.}
      \end{cvitems}
    }
    
  \cventry
    {ACM Electronics Team}
    {Underwater Glider for Real Time Mapping with SensorTag IoT System}
    {}
    {Dec 2016 - Jan 2017}
    {
      \begin{cvitems}
        \item {Accomplished automated glider controlled movement with a ballast system.} 
        \item {Developed obstacle-avoiding feature and algorithm for mapping of environment using MATLAB}
        \item {Interfaced TI CC2650STK SensorTag with Raspberry Pi to retrieve data in real time.}
      \end{cvitems}
    }
 %    \cventry
  %% {\href{https://github.com/akhilesh-k/Intelligent-Alarm}{Intelligent Alarm}}
    %{}
    %{September-November 2016}
    %{
     % \begin{cvitems}
      %\item {Pattern recognizing alarm based on Raspberry Pi which can be set on from a Web UI on local host}
       % \item {Alarm uses a pattern recognizing algorithm on Raspberry Pi to turn off the alarm. The pattern algorithm matches the movement of user with the feed pattern. The implementation has been moved to flask and programming is based on python. .}
      %\end{cvitems}
    %}

\end{cventries}
